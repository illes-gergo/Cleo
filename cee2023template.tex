% convert this file using pdflatex
\documentclass{article}
\RequirePackage[a4paper]{geometry}
\geometry{top=25mm,bottom=25mm,left=25mm,right=25mm,nohead,nofoot,includeheadfoot}
\pagestyle{empty}
\usepackage{float}
\usepackage{mathptmx,graphicx}
\usepackage[english]{babel}
\usepackage{amsmath}
\usepackage{amssymb}
\usepackage{hyphenat}
\usepackage[style=ieee]{biblatex}
\usepackage{csquotes}
\usepackage{caption}
\usepackage[hidelinks]{hyperref}
\addbibresource{refs.bib}
 \captionsetup[figure]{labelfont={bf},name={Fig.},labelsep=space, font=footnotesize, justification=centering}

\AtBeginBibliography{\footnotesize}

\begin{document}
\begin{center}
{\Large\bfseries
Possibility of carbon\hyp{}dioxide pumped terahertz sources\par}
\vspace{3ex}
%{\bfseries
%Author name(s) [10-point type, centred, bolded]\par}
%{\footnotesize\itshape
%Author’s affiliation and full address (8-point type, centred, italicized)\par}
%\vspace{3ex}
{\bfseries
Gergő Illés$^{1}$, Gabit Nazymbekov$^{1}$, György Tóth$^{2}$, János Hebling$^{2}$\par}
{\footnotesize\itshape
1. University of Pécs Ifjúság útja 6, 7624 Pécs, Hungary\\
2. Universit\`a Politecnica delle Marche, IT-60131 Ancona, Italy\par}
\vspace{3ex}
\end{center}
Terahertz generation in nonlinear media is known to be a widely used way to generate single cycle terahertz pulses with great efficiency. Using high pumping intensity in semiconducting materials at the close infrared region is not feasable due to the high multiphoton absorption. \cite{polonyi2016high} As of now, carpon\hyp{}dioxide lasers with sub-picosecond pulse durations are available \cite{polyanskiy2020demonstration}, which can eliminate low-order multiphoton absorption since their central wavelength is approximetly 10 \textmu m.

Numeric 1D+1 calculations were ran with pulse durations between 0.5\hyp{}2.5 ps and intensities from 20\hyp{} to 100 GW/cm$^2$. Our mathematical model took account for the optical rectification, cascading up\hyp{} and downconversion of pump pulse, the self phase modulation of pump pulse \cite{ravi2014limitations} and the second harmonic generated by the pump pulse. The nonlinear media was chosen to be Gallium-Arsenide (GaAs). The result with 1.5 ps pulse duration and 60 GW/cm$^2$ pumping intensity is shown on {Fig.\ref{fig1}}.

%Please be concise in your presentation, highlighting what is novel and
%original about your submission. \textbf{Do not repeat the separate 35
%word abstract.} Simple equations should be included in-line wherever
%possible, whereas more complex expressions should be centred and
%numbered if there are several. Figures should be relevant to the
%submission and preferably centred as shown below. They should be
%placed as close as possible to where they are mentioned in the text.
%Placing subfigures side-by-side is a convenient way to include
%multiple results within the one-page limit. The figures can be
%provided in greyscale or colours. Figure captions should be centred
%beneath figures and in an 8-point font. Figure captions should be
%indented 1 cm on both sides and justified on both right and left
%sides.

%\begin{quotation}
%\vspace{2ex}

%prepage a file figure.png, figure.jpg or figure.pdf to use the following command:
\begin{figure}[H]\centering
\includegraphics[width=\textwidth]{graph1.pdf}
\caption{Results of numeric calculations: a) conversion efficiency, \\b)-c) electric field and spectrum at 3 mm, d)-e) electric field and spectrum at 5 mm}
\label{fig1}
\end{figure}

%\vspace{5cm}

%\vspace{2ex}
%\noindent\footnotesize\textbf{Fig. 1}
%The abbreviation ``Fig.'' (for figure) should appear first, followed
%by the figure number, a period, and then the figure caption.
%\end{quotation}
The results show that carbon\hyp{}dioxide pumped GaAs crystal is capable of generating THz pulses with adequate efficiency and field strength. At 3 mm crystal length the shape of the THz pulse is usable for most applications that require single-cycle THz pulses. The efficiency at 3 mm is 0.1\% and the peak field strength is 200 kV/cm. At 5 mm crystal length the conversion efficiency further increases to 0.9\% however the THz pulse becomes highly modulated.

%References should appear at the end of the article in the order in
%which they are referenced in the body of the paper. The font should be
%8 point, and the references should be aligned left. A suggested format
%for references is given below. Within the main text, references should
%be designated by a number in brackets [1], and they should precede a
%comma or period [2]. Two references cited at once should be included
%together [2,3], separated by a comma, while three or more consecutive
%references should be indicated by the bounding numbers and a dash
%[1–3].
\begin{flushleft}
\printbibliography[title=\normalsize References]
\end{flushleft}

%\setlength\parindent{0pt}\vspace{2ex}
%\textbf{Example References}
%
%\footnotesize
%[1] J. Itatani, D. Zeidler, J. Levesque, D. M. Villeneuve, and P. B.
%Corkum, ``Controlling High Harmonic Generation with Molecular Wave
%Packets,'' Phys. Rev. Lett. \textbf{94}, 123902 (2005). For journal
%articles, authors are listed first, followed by the article’s full
%title in quotes, the journal’s title abbreviation, the volume number
%in bold, page number, and the year in parentheses.
%
%[2] G. P. Agrawal, \textit{Nonlinear Fiber Optics}, 3rd. ed.,
%(Academic Press, Boston, 2001). For citation of a book as a whole:
%authors, followed by title in italics, and publisher, city, and year
%in parenthesis.
%
%[3] R. Kienberger and F. Krausz in, \textit{Few-cycle laser pulse
%generation and its applications}, F.X. Kärtner ed. (Springer Verlag,
%Berlin, 2004). For citation of a book chapter, authors are listed
%first, followed by book title in italics, editors, and publisher,
%city, and year in parenthesis. Chapter number may be added if
%applicable.

% Numerikus számolásokkal megmutattuk hogy co2 lézerrel elő lehet állítani THz-et XY hatásfok mellet ZA csúcstérerőséggel
\end{document}
